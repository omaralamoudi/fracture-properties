\documentclass[12pt]{article}
\title{Project Proposal}
\author{Omar Alamoudi}
\date{March 15, 2019}

\usepackage{mathtools, amssymb}
\usepackage{natbib}
\usepackage[margin=1in]{geometry}

\begin{document}
\maketitle

\section{Objective}
For my research project, I propose an experimental approach to investigate the apparent permeability as a function of in-situ stress, and stress-induced fractuer networks, in low-permeability sedimentary rocks. In addition, since the effective permeability in such rocks is controlled primarily by the connectivity of its fracture networks, understanding the development and evolution of these fracture networks, and how they affect pore fluid transport is of interest. This investigation will be conducted by means of lab experiments, X-ray micro computed tomography imaging (microCT), and mathematical modeling. 

\section{Proposed Project}
In my research, I will be using a conventional tri-axial stress device to deform and fracture rock samples. The samples are enclosed within an X-ray transparent vessel. This allows subjecting the samples to the desired stress conditions while simultaneously acquiring microCT images. In addition to that, we will acquire many different measurements such as: axial strain, axial stress, circumferential strain, confining pressure, etc. We will then used these measurements to describe the mechanical behavior of the tested sample. 

For this class project, I would like to develop a method of characterizing the images produced of the deformed and damaged samples. In our experiment, we intend to acquire a sequence of images over time. The first of which will be of an intact sample, followed by a series of images with the rock deformed and eventually fractured. The goal of this method is to extract geometrical and morphological properties of the fractures. We can then use this information in modeling the fluid flow within the sample as a function of stress.

Since I am in very early stages of my research, I have not acquired any images of deformed nor fractured rock yet. I would therefore intend to start by finding an adequate data set. I would like to start by analyzing microCT images that are taken pre and post failure of a rock sample. Such data set should be sufficient to start the process of analyzing the fractures I am interested in studying.

I am interesting in such project because in previous studies, visual inspection of samples \textbf{during} tri-axial testing experiments was not possible. In fact, visual inspection of samples is limited to two instances, before the experiment, i.e. before the application of stresses or forces, and after conducting the experiment, i.e. after removing the stresses or forces which resulted in sample fracture. Noted observations were then used to infer what occurred in-between these two time instances. Hence, modeling the mechanical fracturing of different samples was only constrained by these inferences. We would like to provide additional constraints to the mechanical modeling of rocks during the tri-axial testing. This is possible due to the advent of microCT imaging. We will perform such experiments and acquire sequential microCT images (different snapshots over time). This will allows us to ``visually" inspect the rock sample during the application of the boundary conditions.

\section{Assumptions and Limitations}
We note here that we will limit our studies to homogeneous rock samples. This is to reduce the uncertainty associated with the complexity of earth materials. In addition, maximum axial stress of about 150 MPa, and maximum confining and pore pressures of about 20 MPa will be used due to the design parameters and the safety of operating our tri-axial testing machine. Finally, keeping in mind the discrete nature of microCT images, image resolution will be limited to the microCT imaging acquisition configuration. 


\end{document} 
